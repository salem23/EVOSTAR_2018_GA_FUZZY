%%%%%%%%%%%%%%%%%%%%%%% file typeinst.tex %%%%%%%%%%%%%%%%%%%%%%%%%
%5
% This is the LaTeX source for the instructions to authors using
% the LaTeX document class 'llncs.cls' for contributions to
% the Lecture Notes in Computer Sciences series.
% http://www.springer.com/lncs       Springer Heidelberg 2006/05/04
%
% It may be used as a template for your own input - copy it
% to a new file with a new name and use it as the basis
% for your article.
%
% NB: the document class 'llncs' has its own and detailed documentation, see
% ftp://ftp.springer.de/data/pubftp/pub/tex/latex/llncs/latex2e/llncsdoc.pdf
%
%%%%%%%%%%%%%%%%%%%%%%%%%%%%%%%%%%%%%%%%%%%%%%%%%%%%%%%%%%%%%%%%%%%


\documentclass[runningheads,a4paper]{llncs}

\usepackage[latin1]{inputenc}
\usepackage{graphicx,color,url}
\usepackage[dvips]{epsfig}
\usepackage{verbatim}
\usepackage{tikz}
\usepackage{subfigure}

\usetikzlibrary{shapes,arrows}
\usetikzlibrary{calc,patterns,snakes,decorations.pathmorphing,decorations.markings}
\usetikzlibrary{positioning}

\providecommand{\tabularnewline}{\\}

\usepackage{amssymb}
\setcounter{tocdepth}{3}
\usepackage{graphicx}

\usepackage{url}
\urldef{\mailsa}\path|salemohammed@gmail.com|
\urldef{\mailsb}\path|amorag@geneura.ugr.es, jjmerelo@gmail.com|
\urldef{\mailsc}\path| fergunet@gmail.com|
\newcommand{\keywords}[1]{\par\addvspace\baselineskip
	\noindent\keywordname\enspace\ignorespaces#1}

\begin{document}
	
\mainmatter  % start of an individual contribution
	
	% first the title is needed
\title{Evolving a TORCS Modular Fuzzy Driver using Genetic Algorithms}
	
	
	%\titlerunning{Driving in TORCS using modular fuzzy controllers}
	
	
	
	
%	\author{Mohammed Salem* \inst{1} \and Antonio Miguel Mora \inst{2}\and Juan Julian Merelo \inst{2} \and Pablo Garc\'ia-S\'anchez \inst{3}}
% Antonio - The submission is blind. ;)
\author{First Author \inst{1}, Second Author \inst{2}, Third Author \inst{3}}	
	
	%\authorrunning{M. Salem et al}
	% (feature abused for this document to repeat the title also on left hand pages)
	
	% the affiliations are given next; don't give your e-mail address
	% unless you accept that it will be published
%	\institute{University of Mascara, Algeria\\
%		\mailsa\\
%		Department of Architecture and Computer Technology
%		University of Granada, Spain\\
%		\mailsb\\
%		Universiy of C\'adiz, Spain
%		\\
%		\mailsc\\
%	}
% Antonio - The submission is blind. ;)

\institute{First Institute
  \and
  Second Institute
  \and
  Third Institute}
% Still need to get the space right - JJ
	%
	% NB: a more complex sample for affiliations and the mapping to the
	% corresponding authors can be found in the file "llncs.dem"
	% (search for the string "\mainmatter" where a contribution starts).
	% "llncs.dem" accompanies the document class "llncs.cls".
	%
	
%	\toctitle{}
%	\tocauthor{}
	\maketitle
	\begin{abstract}
		
This work presents an evolutionary approach to optimize the parameters of a Fuzzy-based autonomous driver for the open simulated car racing game (TORCS). Using evolution, we intend to improve a modular fuzzy agent designed to determine the optimal target speed and steering angle during the race. The challenge in this kind of fuzzy systems is the design of the membership functions, which is usually done through a trial and error process.
In this paper, an adapted real-coded Genetic Algorithm - with two different fitness functions - has been applied to find the best values for these parameters, obtaining a robust design for the TORCS controller.
The evolved drivers were tested and evaluated competing against other TORCS controllers in practice (without rivals) and real races.
%Pablo: compared with other controllers? Explain
% Antonio - Now it's clear I think
The improved fuzzy-controllers yield a very good performance, mainly in tracks that have many turning points, which are, in turn, the most difficult for any
autonomous agent. Thus, this is a real enhancement of the starting fuzzy-controllers which had several difficulties to drive in this kind of circuits.
\end{abstract}

\keywords{Videogames, Fuzzy Controller, TORCS, Steering control, Optimization, Genetic Algorithms}

% These kinds of controllers have a big drawback, which is their membership function parameters are tuned by a trial-error process.

%%%%%%%%%%%%%%%%%%%%%%%%%%%%%%%   INTRODUCTION   %%%%%%%%%%%%%%%%%%%%%%%%%%%%%%%
	%
\section{Introduction}
\label{sec:intro}

Autonomous driving is a very interesting research topic recently
supported by many vehicle manufacturers. The final aim is to create
real self-driving cars that can move in everyday roads and streets or, for
that matter, in a desert or hostile environment. This objective is
complemented with the reduction of fuel consumption and the
maximization of its efficient use, along with the car safety and the
driver comfort in some cases; all these improvements with the implied
constraint that no accidents should take place.  

Optimization in car racing games can be situated in that context, with
solutions obtained there having utility beyond the game itself; this
explains the popularity of car racing games challenges, which are
usually performed using game simulators. The Open Racing Car Simulator
(TORCS) \cite{WebTORCS} is a realistic racing simulator with a
sophisticated engine used for many standalone racing competition
challenges every year. This fact, combined with the large gaming
community and the ability to compare controllers, have made TORCS the
most used simulator in the field of autonomous driving \cite{LFAG,Guadarrama2008,SAES2012,Floreano2004}. 

Many kinds of controllers have been used in this simulator, but one of the most efficient controllers so far are fuzzy-based ones, as they simulate in part the human reasoning when driving \cite{CarRacing_Pelta09,PerezEvolvingFuzzy09,torcs2012}. 

In this line, the authors presented previously an approach in which two specialised fuzzy controllers were combined to decide the car's steering angle and desired speed in every single point (or tick) during a race \cite{evo17_blind}.  

The obtained results were promising, but the performance of the autonomous driver showed some flaws in difficult tracks - those with many curves, or where `external' factors affected the asphalt - and against the most competitive rivals. 

We argue here that the major disadvantage of this approach is that the parameters of the controllers' fuzzy membership functions were defined following a trial/error process in the absence of experts to do so.
Thus, in this work we consider the selection of the best values for these parameters as an optimization problem, so we have applied an evolutionary algorithm to obtain them. Concretely, we propose to apply a real-coded Genetic Algorithm (GA) \cite{realcodedGA1998} with this purpose. The considered approach requires, in addition to a good codification of solutions and selection of operators, the choice of an adequate cost function (fitness), due to the uncertainty and noisiness of the problem itself. So two different proposals have been studied.

The genetic-fuzzy based controller has been evaluated in a practice race - without rivals - first, and then in a real race against different drivers in TORCS.
The obtained results show that the enhanced controllers (one per fitness function) perform both much better than the original fuzzy controller (previously presented) in the practice race, increasing the top speed while they are not damaged, and obtaining a very good lap time.
In addition, they compete very well against tough rivals, getting a rank 2 or 3 in the race with a time close to the winner and a very low received damage.

So, we can conclude from these results that the proposed approach successfully evolves the fuzzy controller membership parameters giving good performances in terms of damage avoidance and race time. This suggest that GAs with the proposed fitness functions are well-suited for finding the best trade-off between the two objectives of any racing controller: damage and speed.


%%%%%%%%%%%%%%%%%%%%%%%%%%%%%%  STATE OF THE ART  %%%%%%%%%%%%%%%%%%%%%%%%%%%%%%
\section{State of the Art}
\label{sec:soa}

Evolutionary algorithms have been used by several researchers in many different ways in the car racing simulator TORCS.

For instance, Loiacono et al \cite{AutogenEvo2011} applied a single-objective and a multi-objective real-coded GA to the automatic generation of tracks for high-end racing games. 
%while \cite{Montecarlo2016} proposes a new framework of applying Monte Carlo Tree Search to the simulated car racing by  building the search tree in a discretized game-state space and then determine the action from the selected target game state to avoid the need to discretize the action space. 

Floreano et al. \cite{Floreano2004}, used a GA for tuning up a neural network which visually recognizes edges, corners and height, resembling strategies observed in simple insects which obtained results that performed equal or better than well trained human drivers tested on the same circuits. 

Another application of evolutionary algorithms was to determine the optimal trajectory of a lap in a known circuit \cite{drivingGA2008}, but this approach suffers from the problem that the obtained trajectory in the evolving process strongly depends on the initial state of the car. 
In the same context, the authors in \cite{GaRaceLine2010} tried to design a novel approach to compute the optimal racing line without any human intervention, using a GA to find the best trade-off between
the minimization of two conflicting objectives: the length and
the curvature of the racing line.

However, definitely, the most prolific area of application of EAs inside TORCS, has been the optimization of autonomous controllers for car driving, i.e. conducting a meta-optimization process.
Thus, EAs have been applied to `refine' the parameters which define the driver's behaviour \cite{ButzCMAES09,SAES2012}, or to improve the structure/architecture of the models \cite{evol,neurone}, working offline, or online (during the game) \cite{TanOnline08,Cardamone_Online_NN}. 

% FUZZY + EAs
Our approach is focused in this line, proposing the application of an off-line genetic algorithm for the improvement of the parameters which determine the behaviour of a controller for TORCS. However, we have focused on a Fuzzy-based model, as it is one of the best options for modelling human-like decisions (and actions). Some others authors have also used this kind of technique in the literature.

For instance, in \cite{Guadarrama2008}, a fuzzy rule-based car controller for a Car Racing Competition was built and tuned with co-evolutionary genetic algorithms. Two fuzzy controllers were designed (acceleration and turning angle). But this approach was applied to a simpler simulator than TORCS which is a more realistic and time-constrained simulator. 
%Pablo: Explain why our approach is better, because TORCS is more complicated
% Mohammed- In TORCS controllers has less time to make a driving action , also it is more realistic than others by considering  traction, friction .....

P�rez et al. presented an evolutionary fuzzy approach for TORCS in \cite{PerezEvolvingFuzzy09}, where they applied EAs for improving fuzzy models to infer the acceleration and turning angle. However, the models were not so specialized as the proposed here, since their controller did not compute the target speed.

Onieva et al.,\cite{LFAG} presented a parametrized modular architecture with a fuzzy system and a GA in the design of fuzzy logic controllers for steering wheel management that can reproduce human driver behaviour, but it did not take the target speed into account unlike our previous controller \cite{evo17_blind} which computed the target speed and the steer with two fuzzy sub-controllers and whose membership functions parameters were defined by trial/error process. 
In this paper, we propose to optimize these parameters using a real coded genetic algorithm aiming to improve the performance of the original fuzzy controller.

%%%%%%%%%%%%%%%%%%%%%%%%%%%%%%%%%%%%%%%%

%Thus, in \cite{SAES2012}, the authors design a competitive driving controller in TORCS  by optimizing the parameters of an autonomous car controller using self adaptive evolutionary strategies (SAESs) which co-evolve solutions and mutation steps for each parameter. 
%Pablo: how we can use this referene to validate ours? Which is our selling-point/differenciation?


% Always "narrate" the SoA relating every paragraph with the previous one
%EAs have been also used to optimize fuzzy set parameters, %Pablo: In TORCS?
%although the maximum values of the sets may not be achieved following this approach \cite{PerezEvolvingFuzzy09}. %Pablo: why?
%In that work, the authors obtain improved fuzzy models to infer the acceleration and turning angle, but they are not specialized as the proposed here. %Pablo: try to explain it better



%%%%%%%%%%%%%%%%%%%%%%%%%%%%%%  TORCS  %%%%%%%%%%%%%%%%%%%%%%%%%%%%%%

\section{TORCS Description}
\label{sec:torcs}

The Open Racing Car Simulator (TORCS) \cite{WebTORCS} is an open source, modern, multi-player, modular and portable racing simulator that allows users to race against computer-controlled opponents. 

Its high degree of modularity and portability, together with the realistic and real-time driving simulation, make it an ideal testbed for artificial intelligence research, as it can be seen in the literature. 
The game offers different types of races from the practical single session to the championship.

There is a large set of sensors \cite{Torcs3} which the car can consider during a race, such as distances to track borders, to rivals, current fuel, current gear, position in the race, speed, or damage, among others.

%%%%%%%%%%%%%%%%%%%%%%	
%	\begin{table}[ht!]
%		{\scriptsize
%			{\centering
%				\begin{tabular}{|p{2cm}|p{3cm}|p{3 cm}|p{3 cm}|}
%					\hline
%					{\textbf{Sensor} }&
%					{\textbf{Name} }&
%					{\textbf{Range} (unit)} &  
%					{\textbf{Data type}}\\ 
%					\hline
%					1 & angle & [-$\pi$,+$\pi$ ] & Double\\ 
%					\hline
%					2 & curLapTime & [0,+$\infty$] (s)	& Double\\ 
%					\hline 
%					3 & damage & [0,+$\infty$)(point)& Double\\ 
%					\hline 
%					4 & distFromStart & [0,+$\infty$) (m)& Double \\ 
%					\hline 
%					5 & distRaced &[0,+$\infty$) (m)& Double\\
%					\hline 
%					
%					6 & focus & [0,200] (m)& Double\\
%					\hline 
%					
%					7 & fuel & [0,+$\infty$) (l)& Double\\
%					
%					\hline
%					8 & gear & \{-1,0,1,.. 6\}g& Integer \\
%					
%					\hline
%					9 & lastLapTime &[0,+1] (s) & Double \\
%					
%					\hline
%					10 & opponents &[0,200] (m)& Double \\
%					
%					\hline
%					10 & racePos & \{1,2,...,N\} & Double \\
%					\hline
%					11 & rpm    & [0,+$\infty$) (rpm)   & Double \\
%					\hline  
%					13 & speedX & (-$\infty$,+$\infty$) (km/h) & Double\\
%					\hline  
%					14 & speedY &(-$\infty$,+$\infty$) (km/h)  & Double\\
%					\hline 
%					15 & speedZ & (-$\infty$,+$\infty$) (km/h) & Double \\
%					
%					\hline
%					16 & track &  [0,200 ] & Double\\ 
%					\hline
%					17 & trackPos & (-$\infty$,+$\infty$) & Double\\
%					
%					\hline
%					
%					18 & wheelSpinVel  & [0,+$\infty$) (rad/s) & Double\\
%					
%					\hline
%					19 & z &  (-$\infty$,+$\infty$) (m) & Double\\
%					
%					\hline
%					
%				\end{tabular}
%			}
%		}
%		\caption{Description of available sensors in TORCS \cite{Torcs3}.}
%		\label{table:TORCS_sensors}
%	\end{table}
	
Every TORCS driver bot is controlled by means of a set of actuators \cite{Torcs3}: the steering wheel `Steer', the accelerator `accel', the brake pedal and the gearbox. 

%Table \ref{table:TORCS_actuators} details the available actions/actuators and their representation.
%	
%	\begin{table}[ht!]
%		{\scriptsize
%			{\centering
%				\begin{tabular}{|p{3cm}|p{3 cm}|p{3 cm}|}
%					\hline
%					
%					{\textbf{Action} }&
%					{\textbf{Range} (unit)} &  
%					{\textbf{Data type}}\\ 
%					\hline
%					Acceleration & [0,+1] & Double\\ 
%					\hline
%					Brake & [0,+1]	& Double\\
%					\hline
%					Gear & -1..0..+6	& Double\\
%					\hline
%					Steer & [-1,+1]	& Double\\
%					\hline
%					Clutch & [-1,+1]	& Double\\
%					\hline
%				\end{tabular}
%			}
%		}
%		\caption{TORCS Actuators \cite{torcs2012}.}
%		\label{table:TORCS_actuators}
%	\end{table}
	
Hence, a controller is a program, which runs inside TORCS, that automatically drives a car. It gets as input information about the current state of the car and its situation on the track (sensors). These collected data are used to decide actions to perform in the next simulation tick.
	
	
%%%%%%%%%%%%%%%%%%%%%%%%%%%  FUZZY CONTROLLER  %%%%%%%%%%%%%%%%%%%%%%%%%%%%

\section{Fuzzy Controller}
\label{sec:fuzzy_controller}

The initial proposed controller \cite{evo17_blind} has the same modular architecture as the simple driver of TORCS, however, the target speed and steering angle are computed by means of two modular and specialised fuzzy sub-controllers, which consider five position sensors. This is the controller which will be improved by means of a GA.

The two sub-controllers are summarized in the following subsections.

%-----------------------------------------------

\subsection{Fuzzy target speed sub-controller}

This controller aims to estimate the optimal target speed of the car, both in straight parts and curves of the track, taking into account two criteria: move as fast as possible and be safe. 

This estimation is based on two general cases: if the car is in a straight line, the target speed will take a maximum value (\textit{maxSpeed} km/h). However, if it is close to a curve, the controller will decrease the current speed to a value included in the interval \textit{[minSpeed, maxSpeed]} km/h.

This fuzzy controller has an output, the speed, and three input values:
\begin{itemize}
	\item Front = Track[9]: front distance to the track border (angle 0�).
	\item M5 = max (Track[8], Track[10]): max distance to the track border in an angle of +5� and -5� with respect to Front.
	\item M10 = max (Track[7], Track[11]): max distance to track border in an angle of +10� and -10�.
\end{itemize}

It is a Mamdani-based fuzzy system \cite{iancu2012} with three trapezoidal Membership Functions (MF) for every input variable. The description of these fuzzy inputs and output are represented in Table \ref{tab:flouevar}.

\begin{table}
		{\scriptsize
	\caption{Fuzzy variables description.}
	\label{tab:flouevar}
	\begin{tabular}{ |p{1.5cm}|p{2cm}|p{2cm}|p{2 cm}|p{1 cm}|p{1.5 cm}|p{1.5 cm}|}
		\hline
		{ \textbf{Variable}}&
		{ \textbf{Range}}&
		{ \textbf{Name}}&  
		{ \textbf{MF}} &
		{ \textbf{Low}} &
		{ \textbf{Medium}}&
		{ \textbf{High}} \\
		\hline
		Input & [0-100] m & Front & trapezoidal & [0-50] & [20-80] & [60-100]
		\\
		\hline
		Input & [0-100] m & M5 & trapezoidal &[0-40] & [10-70] & [50-100] 
		\\
		\hline
		Input & [0-100] m  & M10 & trapezoidal & [0-30] & [20-60] & [50-100]
		\\
		\hline 
		Output & [0-200] m/s & TargetSpeed & singleton & / & / & /
		\\
		\hline 
	\end{tabular} 
}
\end{table}

The base of rules was composed modelling the behaviour of a human expert driver. Thus, this set is designed to maximize the car speed depending on the distance to the track border. The fuzzy rules are:


\begin{itemize}
{\small
	\item \texttt{IF Front is High THEN TargetSpeed is TS1}
	\item \texttt{IF Front is Medium THEN TargetSpeed is TS2}
	\item \texttt{IF Front is Low and M5 is High THEN TargetSpeed is TS3}
	\item \texttt{IF Front is Low and M5 is Medium THEN TargetSpeed is TS4}
	\item \texttt{IF Front is Low and M5 is Low and M10 is High THEN TargetSpeed is TS5}
	\item \texttt{IF Front is Low and M5 is Low and M10 is Medium THEN TargetSpeed is TS6}
	\item \texttt{IF Front is Low and M5 is Low and M10 is Low THEN TargetSpeed is TS7}\\
}

In addition, a crisp rule is added to obtain a maximum value of the target speed when the three input variables are as big as possible:\\
{\small	
\item \texttt{IF Front = MAXDISTSPEED or M5 = MAXDISTSPEED or M10 = MAXDISTSPEED THEN TargetSpeed = MAXSPEED}
}
\end{itemize}

MAXDISTSPEED is the longest possible value for the track sensors, and MAXSPEED is the maximal speed for the specific car. 
The output value is encoded by seven singletons TS1 to TS7, being respectively: 280, 240, 220, 180, 120, 60 and 30.


%***********************************************

\subsection{Fuzzy steering control sub-controller}	

The second fuzzy controller aims to control the steering, estimating and determining the target position of the car. 

The structure of this sub-controller is similar to the speed one, but, obviously, with the steering as output. Thus, the set of sensors considered is the same as in the speed case (in Table \ref{tab:flouevar}).

Then, as general rules: if the car is in a straight line, it will set as target position half width of the race track (central position of the lane). Whereas, if the car is near a right curve, it will approach the path leading to the right, with a space between the car and the border of the track to avoid the loss of control. The same approach is considered if the car is near a left curve.

In order to detect the curves, the controller focuses on the sensor values (M10, M5, and Front). So, if the value on Front sensor is the longest, there is a straight road; whereas if the values of M5 and M10 with positive angles (+5 and +10) are the longest, there is right curve; and the other way round.

The base of rules has been defined again modelling the behaviour of a human driver, so, for this controller:

{\small
\begin{itemize}		
	\item \texttt{IF Front is High THEN steer is S1}
	\item \texttt{IF Front is Medium AND M10 is High THEN  steer is S2}
	\item \texttt{IF Front is Medium AND M10 is Medium AND M5 is Medium THEN steer is S2}
	\item \texttt{IF Front is Medium AND M10 is Medium AND M5 is Low THEN steer is S3}
	\item \texttt{IF Front is Low AND M10 is High THEN steer is S3}
	\item \texttt{IF Front is Low AND M10 is Medium AND M5 is Medium THEN steer is S4}
	\item \texttt{IF Front is Low AND M10 is Medium AND M5 is Low THEN steer is S4}
\end{itemize}	
}

The values for S1 to S4 are respectively: 0, 0.25, 0.5, and 1.
When M10=Track[7] we will take negative values of the steer (steer=-steer).

These controllers were defined with our own criteria, but they could be far from being optimal, so, in the following section we apply a Genetic Algorithm for their improvement.

%%%%%%%%%%%%%%%%%%%%%%%%%%%%  OPTIMISING WITH GAS  %%%%%%%%%%%%%%%%%%%%%%%%%%%%
	
\section{Optimizing the fuzzy controllers with GA}
\label{sec:GA_optimization}

Designing an optimal fuzzy controller for TORCS racing needs a human expert to define the membership functions parameters and the rule base. This expert, even if he exists, could not provide an exact repartition of the fuzzy membership functions values over the universe of discourse. 

This difficulty have led us to move towards the use of Genetic Algorithms (GAs) \cite{GAs_Goldberg89} because of their global exploration characteristic in a complex environment, as this problem plots.

The proposed optimization approach aims to find the optimal parameters of the membership functions of the two sub-controllers previously introduced. 

The followed process is depicted in Figure \ref{fig:ga}, in which, as it can be seen, the GA uses TORCS for the evaluation of every individual during the evolutionary process.

\begin{figure}[!ht]
	\label{fig:ga}
	\begin{center}
		\includegraphics[width=11cm]{fig/flowchart}
	\end{center}
	\caption{Optimization of a fuzzy controller flowchart. The evaluation of an individual is performed by: putting the parameter values on the two sub-controllers, launching a race in TORCS with this configuration, obtaining the resulting values of Damage, Top Speed and mean Lap Time and using these values for the computation of the fitness of the individual.}
\end{figure}	

Indeed, the GA starts by creating the initial population with random values for the parameters in the defined range $[0,100]$. The fitness of each candidate solution is computed by injecting its gene values to the parameters of the membership functions of the two fuzzy sub-controllers. The defined autonomous controller is used to drive a car in a 20 laps race in E-Track5 circuit without opponents, and the results (Top speed, Damage and Mean Lap time) are used to compute the fitness value.


%***********************************************

\subsection{Genetic Algorithm settings}
%
As previously stated, the designed fuzzy controllers have trapezoidal membership functions given by Equation \ref{eq:trapmf}.
In such a controller, fuzzy rules are applied to linguistic terms. These terms, which qualify a linguistic variable, are defined through membership functions, which, in turn, depend on a set of parameters that `describes' their shape (and operation).

The parameters to be optimized are those of all the membership functions that constitute the fuzzy partition of the linguistic variable \cite{ThangG08}.

The input linguistic variables in our problem, \textit{Front, Max5} and \textit{Max10}, are represented by three trapezoidal membership functions (See Table \ref{tab:flouevar}).

A trapezoidal membership function in a finite universe of discourse \textit{[a, b]} can be defined by:

\begin{equation}
\mu_{A}(x)= \left \{
\begin{array}{ll}
\frac{x - x_{1}}{x_{2} - x_{1}},& x_{1} \leq x \leq x_{2}\\
1 , &x_{2} \leq x \leq x_{3}\\
\frac{x_{4} - x}{x_{4} - x_{3}},& x_{3} \leq x \leq x_{4}\\
0        ,& else\\	
\end{array}
\right.
\label{eq:trapmf}
\end{equation}
with:
\begin{equation}
x_{1} \leq x_{2} \leq x_{3} \leq x_{4}
\end{equation}
This MF function is defined by four parameters $x_{1}$, $x_{2}$, $x_{3}$ and $x_{4}$ taking their values in the interval \textit{[a, b]} (See Figure \ref{fig:trapeze}).																			
\begin{figure}[!ht] 
	\begin{center}
		\includegraphics[scale=0.8]{fig/trapese}
		\caption {Trapezoidal MFs}
		\label{fig:trapeze}
	\end{center}
\end{figure}

More generally, a fuzzy partition with \textit{n} trapezoidal membership functions is defined by \textit{2n} variables (\textit{a =} $ x_{1}$,$x_{2} $,. .., $x_{2n} $ \textit {= b})(Equation \ref{eq:e1}). In this case, the representation is given by the figure \ref{fig:at}
\begin{figure}[!ht] 
	\begin{center}
		\includegraphics[scale=0.65]{fig/trapezoidal.png}
		\caption {Trapezoidal-shaped MFs coding}
		\label{fig:at}
	\end{center}
\end{figure}
with:
\begin{equation}
a = x_{1} \leq x_{2} \leq...\leq x_{2n-1} \leq x_{2n}=b 	
\end{equation}		

\begin{equation} 
\begin{tabular}{l}
$\mu_{A1}(x)=  \left \{
\begin{array}{ll}
1, &x_{1} \leq x \leq x_{2}\\
\frac{x_{3} - x}{x_{3} - x_{2}}, &x_{2} \leq x \leq x_{3}\\
0        , &x > x_{3}\\
\end{array} 
\right.$		\\ 	
$\mu_{Ai}(x)= \left \{
\begin{array}{ll} 
0, &x \leq x_{2i-2}\\
\frac{x - x_{2i-2}}{x_{2i-1} - x_{2i-2}}, &x_{2i-2} \leq x \leq x_{2i-1},n=2,...,i-1\\
1, & x_{2i-1} \leq x \leq x_{2i}\\
\frac{x_{2i+1} - x}{x_{2i+1} - x_{2i}},& x_{2i} \leq x \leq x_{2i+1}\\
0  , &x > x_{2i+1}\\
\end{array}  
\right.	$		\\
$\mu_{An}(x)= \left \{
\begin{array}{ll} 
0, &x \leq x_{2n-2}\\
\frac{x - x_{2n-2}}{x_{2n-1} - x_{2n-2}},& x_{2n-2} \leq x \leq x_{2n-1}\\
1 ,& x > x_{2n-1} 
\end{array} 
\right.$\\
\label{eq:e1}
\end{tabular}
\end{equation}
	
As we have just seen, a linguistic variable is represented by a number of parameters that depend both on the number and type of used membership functions  \cite{ThangG08}. Also the choice of coding to use for these different parameters depends both on the desired precision on the values and on their range of values.

When the number of parameters is reduced and their ranges of variations are well defined, a GA with a binary coding is largely sufficient to find their optimal values. On the other hand, if the number of parameters becomes important, and their variation interval is not well known, the real coding is the most appropriate \cite{elsayed13}. 
Since our work requires some precision and the variation interval of each parameter is not well known, we have considered a real coding implementation.

In that GA, every individual is a vector of 18 values/parameters, 6 per variable. Figure \ref {fig:cromosome} illustrates the structure of the chromosome.
%with  :
%% Antonio - Is something missing here? 
%\begin{itemize}
%	\item []$0 = x_{11} \leq x_{12} \leq x_{13} \leq x_{14} \leq x_{15} \leq x_{16} = 100$
%	\item []$0 = x_{21} \leq x_{22} \leq x_{23} \leq x_{24} \leq x_{25} \leq x_{26} = 100$
%	\item []$0 = x_{31} \leq x_{32} \leq x_{33} \leq x_{34} \leq x_{35} \leq x_{36} = 100$
%\end{itemize}
\begin{figure}[!ht]	
	\begin{center}
		\includegraphics[width=13cm]{fig/chromosome2.png}
		\caption{Chromosome description}
		\label{fig:cromosome}	
	\end{center}	
\end{figure}

The initialization of the chromosomes (first population) is performed assigning random values inside a range of variation \cite{GAs_Goldberg89}, in order to start from feasible values \cite{evo17_blind}.
Tournament based selection has been used to elect 
% Antonio - I put 'elect' for not repeating 'selection' and 'select'. ;D
%Mohammed - Sorry, )-:
% Antonio - Don't worry at all. :D
chromosomes as parents for genetic operators, while simple arithmetic two point crossover \cite{crossGA2017} and non uniform mutation \cite{mutation1997} have been chosen, as two of the most contrasted methods in the literature.
% Antonio - Mohammed: we should include a reference to any operator. Moreover, it would be great if we can justify why these operators have been selected (instead of others).
%mohammed- citations added
% Antonio - good. ;)

%***********************************************

\subsection{Fitness definition}

The fitness function for optimizing the structure of a fuzzy controller is highly dependent on the application in which the controller is involved. It is therefore not possible to give a general formulation of this function able to adapt to all the peculiarities of a problem. However, we can specify some rules to choose the best evaluation function \cite{elsayed13}.

The fuzzy controller should aim to: 
\begin{itemize}
	\item  Minimize the damage of the car $damage$.
	\item  Minimize the mean of lap time $LapTime$.
	\item  Maximize the TopSpeed $MaxSpeed$.		
\end{itemize}

From this goals, we can derive two possible fitness functions:

\begin{description}
	\item[Fitness 1:]  
	\begin{equation} \label{fit1}
	\begin{array}{lllll}
	%f_{cout} =  $min$ & \alpha & $. damage$ + & $min$ &\beta & $. temps$ 
	f_{1} =  Min & damage + &\alpha \cdot Min & LapTime 
	\end{array}
	\end{equation}
	\item[Fitness 2:] 
	\begin{equation} \label{fit2}
	\begin{array}{llllll}
	%f_{cout}= min &\alpha & $. dammage$ + & min &\beta & $. temps$ + & min &\gamma & .\frac{1}{vitesse}
	f_{2}= Min & damage + &\alpha \cdot Min & LapTime + & Min & \beta \cdot \frac{1}{TopSpeed}
	\end{array}
	\end{equation}	
\end{description}

Being $\alpha$ and $\beta$ two weighting parameters to prioritize the importance of the different objectives.
To evaluate the candidate controllers during the evolutionary process, we will make each of them compete in a 20 laps practice race in a medium difficulty circuit without rivals. We have omitted the presence of opponents in order to avoid including additional uncertainty sources to the optimization process.
% Antonio - Mohammed: which kind of race (practice or with rivals)? which kind of track? why this track?, which conditions (weather)?, which rivals?
%           This is very important to be mentioned and explained (why have these %           features been selected to correctly evaluate a controller). ;)
% Mohammed- Evaluation is in 20 laps to take the medium of lapstime 
% I  prefered not to include rivals because this  will increase the unknown parameters to take in account in optimization 
% Antonio - Ok, I have written that. ;)

Then, the obtained output values $damage$, $LapTime$ and $TopSpeed$ are collected to compute the corresponding fitness value. As a clarification, $LapTime$ is the average of the 20 laps time.

  		
%%%%%%%%%%%%%%%%%%%%%%%%%%%%  SIMULATION RESULTS  %%%%%%%%%%%%%%%%%%%%%%%%%%%%
	
\section{Simulation Results}
	\label{sec:results}
	
This section is dedicated to the performance evaluation of our fuzzy-genetic controller, called \textit{FGC}.
We first, describe the methodology we have used and next, we present the experimental results of the optimization of the fuzzy-genetic driver.

%***********************************************

\subsection{Simulation settings}
	
	TORCS provides several tracks and cars to choose between. In our case, we have selected the E-Track5 circuit as it is a quite complex one (it has multiple turns). \textit{car1-tbr1} has been selected as the driving car \cite{evo17_blind}. According to previous experiments, this is a fair choice because it is not the fastest car in TORCS (moderate characteristics). This will lead our controller to be prepared to drive in the most usual conditions.

We have evaluated the fuzzy-genetic controller with the two proposed fitness functions, in order to see which one is appropriate to give the best performances. Thus, we have considered the following bots in the experiments:
	
\begin{itemize}
\item \textbf{GFC1}: GA-Fuzzy controller with fitness 1 (Equation \ref{fit1}).
\item \textbf{GFC2}: GA-Fuzzy controller with fitness 2 (Equation \ref{fit2}).
\item \textbf{AD}: Fuzzy controller \cite{evo17_blind}.
\end{itemize}

10 runs of the GA have been conducted in each case, with the configuration shown in Table \ref{tab:GA_config}.

\begin{table}[!ht]	
		\centering
		\caption{GA parameters}
		\label{tab:GA_config}
		\begin{tabular}{|p{3.6cm}|p{3cm}|}
			\hline \textbf{Population size} & 20 \\
			\hline \textbf{Generations} & 50   \\
			\hline \textbf{Crossover rate$\textit{P}_{\textit{c}}$} &  0.7 \\
			\hline \textbf{Mutation rate $\textit{P}_{\textit{m}}$} &  0.3   \\ 		
			\hline          
		\end{tabular}	
\end{table}

The weighting values $\alpha$ and $\beta$ are chosen to be $1$  and $10*MaxSpeed$ respectively where $MaxSpeed $ is the maximum value of speed that \textit{car1-tbr1}  could take ($MaxSpeed=300$) \cite{evo17_blind}, this choice is motivated by the fact to normalize the Top speed values and make them in the same level as other fitness terms.
 The results of these runs are shown in Table \ref{tab:runsresults}.
% Antonio - Comment the obtained values.
\begin{table}[ht]
	\centering\small
	\caption{ Results of 10 runs of GA with the two fitness functions. Best values marked in boldface.}
	{
		\begin{tabular}{|c||c|c|c||c|c|c|c|}
			\hline
			Run &\multicolumn{3}{c||}{Fitness 1}&\multicolumn{4}{c|}{Fitness 2}\\
			\cline{2-8}
			& Min fit. 1 & $LapTime$ & $Damage$ & Min fit. 2& $LapTime$ & $Damage$ & $TopSpeed$\\
			\hline
			1 &46.42& 33.42& 13& 43.8005 & 30.98& 0& 234\\
			
			2 &31.09& 31.09& 0& 51.6614& 29.87& 11& 278\\
			
			3 &\textbf{29.44}& 29.44& 0&\textbf{39.7395} & 29.25& 0& 286\\
			
			4 &31.12& 31.12& 0& 60.5155& 31.64& 16& 233\\
			
			5 &39.63& 31.63&8&  41.7232 & 30.05& 0& 257\\
	            	
			6 &30.21& 30.21& 0&  41.8588 & 30.14& 0& 256\\
			
			7 &29.73& 29.73& 0& 40.3140 &29.12& 0& 268\\
			
			8 &37.29& 31.29&6& 40.8510 & 30.75& 0& 297\\
  			
			9 &34.07& 30.07& 4 & 40.8596 &29.99& 0& 276\\
			
			10&29.88& 29.88& 0& 40.0467&29.63& 0& 288\\
			\hline
			Mean &33.8880 &30.7880 &3.10&   44.1370&  30.1420 &   2.70&267.3000\\ 
			St. Dev.&5.6153&1.1879&4.5814&6.7349&0.7834 &5.8128  & 22.0356\\
			\hline
		\end{tabular}
	}\label{tab:runsresults}
\end{table}

 From The Table, we can see that the $3^{rd}$ run has given the best fitness value for the two considered fitness functions and obtained the minimal Laptime and damage and for the second fitness , the higher TopSpeed in the track.  The fact that the GA  gave the minimum for two fitness functions  in the same run is due to the random initialization of the start population in a favorite region of the space search so the GA has found the optimal solution(We used the same random start population to compare the two fitness functions).\\
 The worst values are obtained by $1^{st}$ Run for fitness 1 and $4^{th}$ run for the second fitness, thee bad values are produced by the damage when the car was in stuck or chocked with the track corner.
 The second fitness has given  higher Top Speed values  which minimize the Laptime and makes the controller operates at higher performances ( late breaking in turns which maximize the Top Speed).
 % Mohammed - I was waiting for Table approval to comment.
%Mohammed - ( Table shows 10 from 20 )  to reduce space.
% Antonio - Mohammed, please compute mean and standard deviation and add them as an additional row on the table. I have marked minimum values in bold.
%Mohammed- Done

% Antonio - Mohammed, the values for Fitness 2 must be wrong. They are the same as LapTime + Damage (as in Fitness 1), however, TopSpeed must be also considered in the calculation. Unless BETA=0, but in that case Fitness 1 = Fitness 2 which will not have sense.

%Mohammed- that's right , I had forgotten \beta in computing fitness from saved file results
%Mohammed - done

% Antonio - Great! :D

The best solution obtained with each fitness function from these runs have been considered to be tested. We represented the resulted membership functions of these two optimal individuals considering the different fitness functions in Figures \ref{fig:mffront}, \ref{fig:mfmax5} and \ref{fig:MFMAX10}.
% These should be examples of results. You can't publish results with a single experiment - JJ
% Antonio - I have commented these are examples and the best results. ;)
%Mohammed -ok

\begin{figure}%
	\centering
	\subfigure[with Fitness 1]{%
		\label{fig:front1}%
		\includegraphics[width=0.6\textwidth,height=3cm]{fig/MFFRONT}}%
	\subfigure[with Fitness 2]{%
		\label{fig:front2}%
		\includegraphics[width=0.6\textwidth,height=3cm]{fig/MFFRONT2}}%
	\caption{Front input MFs with GA}
	\label{fig:mffront}
\end{figure}

\begin{figure}%
	\centering
	\subfigure[with Fitness 1]{%
		\label{fig:fmax51}%
		\includegraphics[width=0.6\textwidth,height=3cm]{fig/MFFMAX5}}%
	\subfigure[with Fitness 2]{%
		\label{fig:fmax52}%
		\includegraphics[width=0.6\textwidth,height=3cm]{fig/MFFMAX52}}%
	\caption{Max5 input MFs with GA}
	\label{fig:mfmax5}%
\end{figure}

\begin{figure}%
	\centering
	\subfigure[with Fitness 1]{%
		\label{fig:MFMAX101}%
		\includegraphics[width=0.6\textwidth,height=3cm]{fig/MFMAX10}}%
	\subfigure[with Fitness 2]{%
		\label{fig:MFMAX102}%
		\includegraphics[width=0.6\textwidth,height=3cm]{fig/MFMAX102}}%
	\caption{Max10 input MFs with GA}
		\label{fig:MFMAX10}%
\end{figure}

The shapes of the obtained memberships are completely different from those obtained by Trial/Error in the previous work \cite{evo17_blind} where the Medium linguistic variable of the new functions has bigger range. This makes the controller very sensitive to the middle distances of the inputs, like for a real driver who considers most of the cases the car distance from the borders in that range.
The other remark from the obtained  membership functions is the dimension of the common range between the LOW and MEDIUM, which provide a higher diversity in the output values.
% Antonio - Yes, but this can be seen in the figure. I meant you should say something interesting as if the new MFs give a higher relevance to one of the ranges (LOW, MEDIUM, HIGH), and what does this mean from the point of view of the controller. ;D
%Mohammed- working on it
% Antonio - ok, it's fine now. ;)

%%%%%%%%%%%%%%%%%%%%%%%%%%%%%%%%%%%%%%%%%%%%%%%%%%%%%

\subsection{GA-Fuzzy controller in practice race} 

The genetic based fuzzy controllers obtained from the last section are tested in a practice race where these two evolved controllers, named $GFC1$ and $GFC2$, together with the initial fuzzy controller, $AD$, will run each one for 20 laps in E-Track5 circuit (the one used during the evolution) and after they will be tested also in a practice race in an unknown track for them as E-Road. The obtained results are presented in Table \ref{resultat20}.

\begin{table}[!ht]
	\centering
	\caption{Results of the three controllers in a 20 laps practice race}
	\label{resultat20}
	\begin{tabular}{|p{3cm}|c|c|c|}
		\hline
		\multicolumn{4}{|c|}{\textbf{E-Track 5}}  \\
		\hline \textbf{Results} & \textbf{AD} & \textbf{GFC1} & \textbf{GFC2}  \\
		\hline Best Lap Time         & 29:70 & 30:03 & 29:50 \\
		\hline Top Speed          & 209 & 216 & 216\\
		\hline Min Speed          & 168 & 148 & 182 \\
		\hline Last Lap Time       & 29:79 &  30:03 & 29:94\\
		\hline Damage          & 936 & 0 & 0\\
		\hline 
		\hline
		\multicolumn{4}{|c|}{\textbf{E-Road}}  \\ 
		\hline \textbf{Results} & \textbf{AD} & \textbf{GFC1} & \textbf{GFC2}  \\
		\hline Best Lap Time         & 02:31:71    & 02:26:72       & 02:26:54  \\
		\hline Top Speed& 206& 205 &  208 \\
		\hline Min Speed          & 30          & 39             & 37  \\
		\hline Last Lap Time       & 03:12:79    &  02:42:26      &  02:33:36 \\
		\hline Damage          &   0         & 0              & 0  \\
		\hline 
	\end{tabular}
\end{table}
% Antonio - Mohammed, please check the values of the first track (E-Track 5) as the best lap and the last lap times are always the same (for the 3 controllers).
%Mohammed- The second controller  obtained the best lap in the last one, for the other , I corrected it
% Antonio - Great. :)

From the table, we can see that the fuzzy controllers optimized by the GA have given the best results, minimizing the global race time and damages ($0$ in the two GA-based fuzzy controllers), while the AD controller has finished the practice race with many damages.
The consideration of a new track for the controllers as E-Road (quite long and difficult), has proved their value in the adaptation to other tracks different from the one used for `training' (optimizing).

GFC2 controller has run with a higher Speed (considering overall Top Speed and Min Speed) than GFC1 in the two tracks. This is a positive consequence of the inclusion of $TopSpeed$ variable in the fitness computation so the GA based fuzzy controller has optimized the speed of the car due to early braking and detection of turns and their curving angles. This ability  of the GA-fuzzy controller collaborates to overall race time minimization and the final ranking.

% Antonio - add more comments on the results
% Mohammed- done
According to these results, we can conclude that GFC2 is the best controller.


%%%%%%%%%%%%%%%%%%%%%%%%%%%%%%%%%%%%%%%%%%%%%%%%%%%%%

\subsection{GA-Fuzzy controllers in a real race}

Every fuzzy controller has been tested separately from the others in a real race against five standard controllers from each team integrated with TORCS. Tables \ref{tab:gfc1real}, \ref{tab:gfc2real} and \ref{tab:adreal} illustrate their performance in a 5 laps real race. 

\begin{table}[!ht]
		\centering
	\caption{Results of GFC1 in a real race (5 laps)}
	\label{tab:gfc1real}
	\begin{tabular}{|p{2.3cm}|p{1.5 cm}|p{1.75 cm}|p{1.5 cm}|p{1.75 cm}|p{1.5 cm}|p{1.5 cm}|}
		\hline \textbf{E-Track5} &   \textbf{GFC1} & \textbf{berwin 10} & \textbf{bt 3} &\textbf{damned 2} & \textbf{inferno 5} & \textbf{tita 10}  \\
		\hline \textbf{Ranking} & 3/6&4/6&1/6&5/6&2/6&6/6\\			
		\hline \textbf{Race Time}	& 02:30:74\newline+35:70&02:30:74\newline +1 lap &02:30:74&02:30:74\newline +1 lap &02:30:74\newline+12:13&02:30:74\newline+1 lap\\	
		\hline \textbf{Best Lap}&35:65& 36:39&28:57&36:83&30:53&35:39\\	
		\hline \textbf{Max Speed}& 196&202&231&192&226&202\\	
		\hline \textbf{Damages}& 0&0&0&603&0&471 \\	
		\hline 
	\end{tabular}
\end{table}
\begin{table}[!ht]
	\centering
	\caption{Results of GFC2 in a real race (5 laps)}
	\label{tab:gfc2real}
	\begin{tabular}{|p{2.3cm}|p{1.5 cm}|p{1.75 cm}|p{1.5 cm}|p{1.75 cm}|p{1.5 cm}|p{1.5 cm}|}
		\hline \textbf{E-Track5} & \textbf{GFC2}&\textbf{berwin 10} & \textbf{bt 3} &\textbf{damned 2} & \textbf{inferno 5} & \textbf{tita 10}  \\
		\hline \textbf{Ranking} & 2/6&4/6&1/6&6/6&3/6&5/6\\			
		\hline \textbf{Race Time}	& 02:30:83\newline +03:99&  02:30:83\newline+1 lap&02:30:83&02:30:83\newline+1 lap&02:30:83\newline+08:35&02:30:83\newline+1 lap\\	
		\hline \textbf{Best Time}& 29:82 &36:38&28:35&37:04&30:53&36:00\\	
		\hline \textbf{Max Speed}& 214&202&230&188&226&204\\	
		\hline \textbf{Damage}& 0& 0 & 343&1230&0&668\\	
		\hline 
	\end{tabular}
\end{table}
\begin{table}[!ht]
		\centering
	\caption{Results of AD in a real race}
	\label{tab:adreal}
	\begin{tabular}{|p{2.3 cm}|p{1.5 cm}|p{1.75 cm}|p{1.5 cm}|p{1.75 cm}|p{1.5 cm}|p{1.5 cm}|}
		\hline \textbf{E-Track5} & \textbf{AD} & \textbf{berwin 10} & \textbf{bt 3} &\textbf{damned 2} & \textbf{inferno 5} & \textbf{tita 10}  \\
		\hline \textbf{Ranking} & 6/6&5/6&1/6&4/6&2/6&3/6\\			
		\hline \textbf{Race Time}	& 02:31:83\newline+5laps&02:31:83\newline+1 lap&02:31:83&02:31:83\newline+1 lap&02:31:83\newline+21:32&02:31:83\newline+33:43 \\	
		\hline \textbf{Best Time}& 00:00&37:25&28:60&36:28&31:47&35:84\\	
		\hline \textbf{Max Speed}& 111&202&230&189&225&202\\	
		\hline \textbf{Damage}& 10786&465&5&0&2394&0 \\	
		\hline 
	\end{tabular}
\end{table}

We note that the obtained ranking by the controllers $GFC1$,   $GFC2$ was third and second respectively. 
Regarding the race time, GFC2 has a better time than GFC1 while GFC1 and GFC2 finished the race without recording any damage.
On the other hand, the AD controller was not able to avoid high damages in the track.
These results are a confirmation of the good optimization done by the genetic algorithms and mainly  when the Top Speed was considered in the fitness. 
The obtained results in real race with opponents from tough teams from TORCS are encouraging even the optimization process was in practice races.
This good adaptation of the proposed controller in races with rivals is due to the fact the modular fuzzy controller takes into consideration the presence of rivals in the track\cite{evo17_blind}. 
The enhancement  of that driver by the optimal values of the membership function values,  allows it to detect and overtake the opponents with no damage or stuck.
 
% Antonio - TO DO: improve the analysis of results.
%Mohammed- done

%%%%%%%%%%%%%%%%%%%%%%%%%%%%  CONCLUSIONS  %%%%%%%%%%%%%%%%%%%%%%%%%%%%
\section{Conclusions and Future Work} 
\label{sec:conclusions}

In this work, we presented a Genetic Algorithm implementation to optimize and improve an existing fuzzy driver for TORCS simulator (AD) \cite{evo17_blind}, which combines two sub-controllers, one to calculate the target speed and the other for the direction (steer).

In the evolutionary approach, we tested two fitness functions which use the results obtained in a race conducted by the controller being optimized. 
The first one considers the average lap time and the car damage; the second uses these two factors and also the top speed reached.

The fuzzy-genetic controllers obtained were compared with the original AD controller whose parameters are determined empirically. The comparison was made first without opponents and then, with several cars from the TORCS teams in realistic races.

The obtained results are very promising since the optimized controllers overcome the original in the practice races, and also they were ranked among the first ones in the evaluation races, with the minimum of damage.

Nevertheless, these results can be improved by extending the evaluation of population controllers in the Genetic algorithm to other tracks and not just one, to allow the elected controller to practice effectively in multiple tracks and in real-life situations.
Moreover, we could also try to generate, optimize and tune automatically the rule base of the fuzzy controller by means of a Genetic Programming algorithm.


\section*{Acknowledgments}

% This work has been supported in part by: Ministerio espa\~{n}ol de
% Econom\'{\i}a y Competitividad under project TIN2014-56494-C4-3-P
% (UGR-EPHEMECH).
Hidden for double blind\\
Taking this much space.

	\bibliographystyle{splncs03}
	\bibliography{fuzzy_torcs}

\end{document}
